\documentclass[10pt]{beamer}
\usetheme[
%%% option passed to the outer theme
%    progressstyle=fixedCircCnt,   % fixedCircCnt, movingCircCnt (moving is deault)
  ]{Feather}
  
% If you want to change the colors of the various elements in the theme, edit and uncomment the following lines

% Change the bar colors:
%\setbeamercolor{Feather}{fg=red!20,bg=red}

% Change the color of the structural elements:
%\setbeamercolor{structure}{fg=red}

% Change the frame title text color:
%\setbeamercolor{frametitle}{fg=blue}

% Change the normal text color background:
%\setbeamercolor{normal text}{fg=black,bg=gray!10}

%-------------------------------------------------------
% INCLUDE PACKAGES
%-------------------------------------------------------

\usepackage[utf8]{inputenc}
\usepackage[english]{babel}
\usepackage[T1]{fontenc}
\usepackage{helvet}

%-------------------------------------------------------
% DEFFINING AND REDEFINING COMMANDS
%-------------------------------------------------------

% colored hyperlinks
\newcommand{\chref}[2]{
  \href{#1}{{\usebeamercolor[bg]{Feather}#2}}
}

%-------------------------------------------------------
% INFORMATION IN THE TITLE PAGE
%-------------------------------------------------------

\title[] % [] is optional - is placed on the bottom of the sidebar on every slide
{ % is placed on the title page
      \textbf{Sistemas de Recomendación}
}

\subtitle[Sistemas de Recomendación]
{
      \textbf{v. 1.0.0}
}

\author[Uayeb Caballero]
{      Uayeb Caballero \\
      {\ttfamily uayeb.caballero@gmail.com}
}

\institute[]
{
      Universidad Nacional Autonoma de Honduras
  
  %there must be an empty line above this line - otherwise some unwanted space is added between the university and the country (I do not know why;( )
}

\date{\today}

%-------------------------------------------------------
% THE BODY OF THE PRESENTATION
%-------------------------------------------------------

\begin{document}

%-------------------------------------------------------
% THE TITLEPAGE
%-------------------------------------------------------

{\1% % this is the name of the PDF file for the background
\begin{frame}[plain,noframenumbering] % the plain option removes the header from the title page, noframenumbering removes the numbering of this frame only
  \titlepage % call the title page information from above
\end{frame}}

%-------------------------------------------------------
\section{Introduction}
%-------------------------------------------------------

\begin{frame}{Introduction}
%-------------------------------------------------------
Los sistemas de recomendaciones son herramientas que generan recomendaciones sobre un determinado objeto de estudio, 
a partir de las preferencias y opiniones dadas por los usuarios. El uso de estos sistemas se está poniendo cada vez más de
moda en Internet debido a que son muy útiles para evaluar y filtrar la gran cantidad de información disponible en la
Web con objeto de asistir a los usuarios en sus procesos de búsqueda y recuperación de información. 
En este trabajo realizaremos una revisión de las características y aspectos fundamentales relacionados con el diseño, 
implementación y estructura de los sistemas de recomendaciones analizando distintas propuestas que han ido apareciendo 
en la literatura al respecto.
\end{frame}


%-------------------------------------------------------
\section{Collaborative Filtering}
%-------------------------------------------------------

\begin{frame}{Collaborative Filtering}
%-------------------------------------------------------
El Filtrado colaborativo (FC) es una técnica utilizada por algunos sistemas recomendadores. En general,
el filtrado colaborativo es el proceso de filtrado de información o modelos, que usa técnicas que implican 
la colaboración entre múltiples agentes, fuentes de datos, etc.  Las aplicaciones del filtrado colaborativo 
suelen incluir conjuntos de datos muy grandes. Los métodos de filtrado colaborativo se han aplicado a muchos 
tipos de datos, incluyendo la detección y control de datos (como en la exploración mineral, sensores 
ambientales en áreas grandes o sensores múltiples, datos financieros) tales como instituciones de 
servicios financieros que integran diversas fuentes financieras, o en formato de comercio electrónico y 
aplicaciones web 2.0 donde el foco está en los datos del usuario, etc. Esta discusión se centra en el 
filtrado colaborativo para datos de usuario, aunque algunos de los métodos y enfoques pueden 
aplicarse a otras aplicaciones.


\end{frame}


%-------------------------------------------------------
\section{Índice Jaccard}
%-------------------------------------------------------

\begin{frame}{Índice Jaccard}
%-------------------------------------------------------
El índice de Jaccard ( IJ ) o coeficiente de Jaccard ( IJ ) mide el grado de similitud entre dos conjuntos, sea cual sea el tipo de elementos.
$\newline$
La formulación es la siguiente:
$\newline$
J(A,B) = |A ${\cap}$ B| / |A ${\cup}$ B|

\end{frame}


%-------------------------------------------------------
\section{Basado en memoria}
%-------------------------------------------------------

\begin{frame}{Basado en memoria}
%-------------------------------------------------------
Este mecanismo utiliza los datos de las evaluaciones de los usuarios para calcular la similitud entre los usuarios o 
elementos. Esto se utiliza para hacer recomendaciones. Este fue de los primeros mecanismos y se usa en muchos 
sistemas comerciales.

${\displaystyle \operatorname {sim} (x,y)=\cos({\vec {x}},{\vec {y}})={\frac {{\vec {x}}\cdot {\vec {y}}}{||{\vec {x}}||\times ||{\vec {y}}||}}={\frac {\sum \limits _{i\in I_{xy}}r_{x,i}r_{y,i}}{{\sqrt {\sum \limits _{i\in I_{x}}r_{x,i}^{2}}}{\sqrt {\sum \limits _{i\in I_{y}}r_{y,i}^{2}}}}}}$

\end{frame}


%-------------------------------------------------------
\section{Reducción de Dimensiones}
%-------------------------------------------------------

\begin{frame}{Reducción de Dimensiones}
%-------------------------------------------------------
Para conjuntos de datos de alta dimensión (es decir, con número de dimensiones más de 10), reducción de la dimensión 
se realiza generalmente antes de la aplicación de un K-vecinos más cercanos (k-NN) con el fin de evitar los 
efectos de la maldición de la dimensionalidad. 
$\newline$
Extracción de características y la reducción de la dimensión se puede combinar en un solo paso utilizando 
análisis de componentes principales (PCA), análisis discriminante lineal (LDA), o análisis de la correlación 
canónica (CCA) técnicas como un paso pre-procesamiento seguido por la agrupación de K-NN en vectores de 
características en el espacio reducido dimensión. En aprendizaje automático este proceso de pocas dimensiones 
también se llama incrustar.
\end{frame}


%-------------------------------------------------------
\section{Descomposición en valores singulares}
%-------------------------------------------------------

\begin{frame}{Descomposición en valores singulares}
%-------------------------------------------------------
En álgebra lineal, la descomposición en valores singulares de una matriz real o compleja es una factorización de 
la misma con muchas aplicaciones en estadística y otras disciplinas.

Dada una matriz real ${\displaystyle A\in \Re ^{m\times n}}$, los autovalores de la matriz cuadrada, 
simétrica y semidefinida positiva ${\displaystyle A^{T}A\in \Re ^{n\times n}}$ son siempre reales y mayores o iguales a cero. 
Teniendo en cuenta el producto interno canónico vemos que:

$\newline$

${\displaystyle \left(A^{T}A\right)^{T}=A^{T}\left(A^{T}\right)^{T}=A^{T}A}$. O sea que es simétrica
$\newline$
${\displaystyle (Ax,Ax)=x^{T}A^{T}Ax=||Ax||^{2}\geq 0}$ es decir ${\displaystyle A^{T}A\,}$ es semidefinida positiva, 
es decir, todos sus autovalores son mayores o iguales a cero.
$\newline$

Si ${\displaystyle \lambda _{i}\,}$  es el i-ésimo autovalor asociado al i-ésimo autovector, 
entonces ${\displaystyle \lambda _{i}\in \Re }$. Esto es una propiedad de las matrices simétricas.
\end{frame}


%-------------------------------------------------------
\section{Conclusión}
%-------------------------------------------------------

\begin{frame}{Conclusión}
%-------------------------------------------------------
La cantidad de algoritmos para construir sistemas de recomendacion son incontables, hoy en dia con el constante crecimiento de Internet
y con las distintas necesidades para retener los usuarios, los recommendation engines deben de ser mas precisos hasta el punto de 
dar la experiencia de personalizacion por usuario. tener este tipo de paradigmas hace que el costo computacional sea elevado en el 
caso de querer implementar tecnicas en base a filtros colaborativos, por lo que para datasets pequeños se recomienda su uso. lo bueno
de este tipo de tecnicas es que su interpretabilidad es mas rapida. para el caso de datasets muy grandes se recomiendan tecnicas que involucran 
reduccion de dimensiones como el caso de Descomposicion Espectral, SVD o ACP. Estos por su tecnica son mas rapidos de procesar
pero mas dificiles de interpretar. 
\end{frame}



\end{document}